%%%%%%%%%%%%%%%%%%%%%%%%%%%%%%%%%%%%%%%%%%%%%%%%%%%%%%%%%%%%%%%%%%%%%%%%%%%%%
%$HeadURL: https://www.mech.kth.se/svn/simson/trunk/doc/simson-svn-guide.tex $
%$LastChangedDate: 2007-12-18 23:54:12 +0100 (Tue, 18 Dec 2007) $
%$LastChangedBy: mattias@MECH.KTH.SE $
%$LastChangedRevision: 1074 $
%%%%%%%%%%%%%%%%%%%%%%%%%%%%%%%%%%%%%%%%%%%%%%%%%%%%%%%%%%%%%%%%%%%%%%%%%%%%%
%
% To compile this document do the following:
%
%   pdflatex simson-svn-guide
%   bibtex simson-svn-guide
%   pdflatex simson-svn-guide
%   pdflatex simson-svn-guide
%
%%%%%%%%%%%%%%%%%%%%%%%%%%%%%%%%%%%%%%%%%%%%%%%%%%%%%%%%%%%%%%%%%%%%%%%%%%%%%

\documentclass[10pt,a4paper]{simson}
\usepackage[swedish,english]{babel}
\usepackage{graphicx}
\usepackage{amsmath}
\usepackage{natbib}
\usepackage[latin1]{inputenc}
\usepackage[colorlinks=true, pdfstartview=FitV, linkcolor=blue, citecolor=blue, urlcolor=blue, pdfkeywords={DNS, Simson, subversion, svn}, pdfauthor={}, pdftitle={User Guide. Produced from $Rev: 1074 $}]{hyperref}

% Set graphics path
\graphicspath{{./figures/}}

\title{\textbf{SIMSON}\\Subversion Quickstart\\ $Rev: 1074 $}

\author{}

\kthno{}
\kthcoverpicture{figures/freestream_front}
\kthinnerbackone{The front page image is a visualization of laminar-turbulent transition induced by ambient free-stream turbulence convected above a flat plate, \emph{i.e.}\ so-called bypass transition \citep{Brandt-Schlatter-Henningson:2004}. The large-eddy simulation used to generate the image \citep{schlatter_brandt_henningson_abstract_2006} is performed with \esoft{Simson} using the ADM-RT subgrid-scale model \citep{schlatter_stolz_kleiser_2004}, further postprocessed using the program \esoft{lambda2} and rendered using \esoft{OpenDX}. Low and high speed streaks are visualized with blue and red isocontours, respectively; green and yellow isocontours indicate the $\lambda_2$ vortex identification criterion \citep{jeong_hussain_1995}. The flow is from lower left to upper right.}
\kthinnerbacktwo{This user guide is compiled from $Rev: 1074 $.}

\begin{document}

\maketitle

\tableofcontents*


% ####################################################################
\chapter{Introduction}
% ####################################################################
\setcounter{page}{1}\pagenumbering{arabic} To ease the development of
\esoft{Simson} the version control system \esoft{Subversion} has been
used. \esoft{Subversion} works by keeping all source files including
their history in a database called a repository where the data is
stored in an efficient manner.

Each developer has their own private working copy of the files. Each
working copy directory contains a special directory named \epath{.svn}
where some administrative files are stored. The developer decides when
and how to synchronize the working copy with the repository.

Read and write access to the repository can be specified to certain
directories for certain users if needed. This allows developers,
responsible for different parts of the code, to work independently.

In this report follows a short description on how to use
\esoft{Subversion} and the most common commands. The entire
\esoft{Subversion} book/manual can be found in \cite{svn-book}.


% ####################################################################
\chapter{Getting started}
% ####################################################################

% ====================================================================
\section{Creating a working copy root directory}
% ====================================================================
The following command creates a new working directory \epath{simson}
and puts the latest source code from the \esoft{Simson} trunk
repository there:
\begin{verbatim}
  svn checkout https://www2.mech.kth.se/svn/simson/trunk simson
\end{verbatim}
The \ecommand{checkout} subcommand is only used when you want to
create a new private working copy from scratch.

Now you can edit the files in the \epath{simson} directory. Changes in
this directory do not affect other developers. The local \efile{.svn}
directories contain administrative files that \esoft{Subversion} uses
to keep track of your files and should not be changed.


% ====================================================================
\section{The most commonly used Subversion subcommands}
% ====================================================================

% --------------------------------------------------------------------
\subsection{General work flow}
% --------------------------------------------------------------------
This is the most often used \esoft{Subversion} commands and the
general work flow goes through the following items. Each command is
described in the following sections.

Update your working copy:
\begin{itemize}
\item svn update
\end{itemize}

Make changes:
\begin{itemize}
\item svn add
\item svn delete
\item svn copy
\item svn move
\end{itemize}

Examine your changes:
\begin{itemize}
\item svn status
\item svn diff
\item svn revert
\item tkdiff (svn-aware graphical diff frontend)
\end{itemize}

Merge others' changes into your working copy:
\begin{itemize}
\item svn update (IMPORTANT!)
\item svn resolved
\end{itemize}

Commit your changes:
\begin{itemize}
\item svn commit
\end{itemize}

Note that it is crucial that you update your copy of the repository
before you commit your changes. Otherwise changes committed by others
between the previous update of your copy and now could be lost.


% --------------------------------------------------------------------
\subsection{Getting help}
% --------------------------------------------------------------------
If you want a list of all \esoft{Subversion} subcommands, use
\ecommand{help}:
\begin{verbatim}
  svn help
\end{verbatim}
Type \ecommand{svn help subcommand} for help on a specific subcommand.

Most subcommands take file and/or directory arguments, recursing on
the directories. If no arguments are supplied, it will recurse on the
current directory (inclusive) by default. Available subcommands are
(aliases are given within parentheses):
\begin{verbatim}
  add
  blame (praise, annotate, ann)
  cat
  checkout (co)
  cleanup
  commit (ci)
  copy (cp)
  delete (del, remove, rm)
  diff (di)
  export
  help (?, h)
  import
  info
  list (ls)
  log
  merge
  mkdir
  move (mv, rename, ren)
  propdel (pdel, pd)
  propedit (pedit, pe)
  propget (pget, pg)
  proplist (plist, pl)
  propset (pset, ps)
  resolved
  revert
  status (stat, st)
  switch (sw)
  update (up)
\end{verbatim}


% --------------------------------------------------------------------
\subsection{Examining and comparing working copy with the repository}
% --------------------------------------------------------------------
The most commonly used subcommand reports what files you have modified
in your working directory among other things:
\begin{verbatim}
  svn status
\end{verbatim}
The \ecommand{status} subcommand and many other subcommands works
recursively on directories (by default the current working
directory). If you want the status of a named directory or a single
file you add the name of that file or directory. For instance:
\begin{verbatim}
  svn status mydirectory/myfile.f
\end{verbatim}
There is an important extra flag to the \ecommand{status} subcommand:
\ecommand{-u}. The \ecommand{-u} flag is used to display pending
updates from the repository.
\begin{verbatim}
  svn status -u -v mydirectory/myfile.f
\end{verbatim}
To view the history of a file or directory, use the \ecommand{log}
subcommand. This shows who changed the file, when they did it and the
log message they provided to describe their changes.
\begin{verbatim}
  svn log
\end{verbatim}
The \ecommand{diff} subcommand examines the difference between your
working copy and the repository more in detail. The output is similar
to the output of the Unix command \ecommand{diff}; \emph{e.g.}
\begin{verbatim}
  svn diff afortranfile.f
\end{verbatim}
A graphical frontend to \ecommand{diff} is the freely available \ecommand{tkdiff}, which is svn-aware. For example,
\begin{verbatim}
  tkdiff afortranfile.f
\end{verbatim}
displays all local changes of the given file since the last update, whereas
\begin{verbatim}
  tkdiff -r100 afortranfile.f
\end{verbatim}
shows all the changes compared to revision 100 in the repository.


% --------------------------------------------------------------------
\subsection{Undoing changes to the working copy}
% --------------------------------------------------------------------
If you want to undo your changes to some file or directory in your
working copy there is a convenient way. Use \ecommand{revert}:
\begin{verbatim}
  svn revert myfile.f
\end{verbatim}
After the \ecommand{revert} subcommand has finished the working copy
has the same state as when you did \ecommand{update} or
\ecommand{checkout} the last time.

The \ecommand{revert} subcommand may also be applied to several files
at once, or an entire directory. Because the \ecommand{revert}
subcommand reverts to the mirror-copy kept in the \efile{.svn}
directories it works even when there is no network connection to the
repository, such as when working with a laptop.


% --------------------------------------------------------------------
\subsection{Updating working copy}
% --------------------------------------------------------------------
The \ecommand{update} subcommand updates your working copy and the
\efile{.svn} directories with the latest changes to the repository.

All files in your current working directory and below are updated. If
you have also changed the contents of your working copy the repository
changes are merged with your changes.
\begin{verbatim}
  svn update
\end{verbatim}
Again, you may update a single file or directory by naming it:
\begin{verbatim}
  svn update mydirectory/myfile.f
\end{verbatim}
Note that you can run \ecommand{status -u} to get a list of all
updates available in the repository. This is a good way to predict
what \ecommand{update} will do.

The \ecommand{update} subcommand is also used when you want to
retrieve an old version of some file(s) from the repository; \emph{e.g.}\ to
get the repository version of \ecommand{myfile.f} as it was on the
6:th of October 2004 you write:
\begin{verbatim}
  svn update -r '{20041006}' mydirectory/myfile.f
\end{verbatim}
In some cases \ecommand{update} will fail to merge changes to the
repository with your local changes. If \ecommand{update} or
\ecommand{status} lists a file with a ``C'' in front this means that
you need to merge the reported file(s) manually.


% --------------------------------------------------------------------
\subsection{Committing modified files into the repository}
% --------------------------------------------------------------------
The \ecommand{commit} subcommand updates the repository with your
changes.
\begin{verbatim}
  svn commit -m 'New adiabatic boundary condition added'
\end{verbatim}
This commits all files in the current directory and below
recursively. You may also commit a single file or directory by naming
it:
\begin{verbatim}
  svn commit mydirectory/somefile.f -m 'Fixed memory leak problem'
\end{verbatim}
Note that \ecommand{commit} will safely fail if you do not have
write access to all corresponding files in the repository.


% ====================================================================
\section{Other useful subcommands}
% ====================================================================

% --------------------------------------------------------------------
\subsection{Commands for moving, removing, and adding files}
% --------------------------------------------------------------------
If you add or delete files from your working copy you need to
explicitly tell \esoft{Subversion} about this. The subcommands
\ecommand{add}, \ecommand{move}, \ecommand{copy} and \ecommand{delete}
are used for this. See \ecommand{svn help xxx} for help on these
subcommands. The \ecommand{rename} subcommand is an alias for
\ecommand{move}.

As usual, the repository is not changed when you apply these commands
on your working copy. When you do \ecommand{svn commit} the repository
changed.


% --------------------------------------------------------------------
\subsection{Adding the keywords to a new file}
% --------------------------------------------------------------------
\esoft{Subversion} can substitute certain information directly into
the files. This is done by putting keywords inside the file. When you
add a new file for version control, put the following lines in the
beginning of that file:
\begin{verbatim}
  $HeadURL: https://www.mech.kth.se/svn/simson/trunk/doc/simson-svn-guide.tex $
  $LastChangedDate: 2007-12-18 23:54:12 +0100 (Tue, 18 Dec 2007) $
  $LastChangedBy: mattias@MECH.KTH.SE $
  $LastChangedRevision: 1074 $
\end{verbatim}
You must tell \esoft{Subversion} which keywords to look for. This is
done through the following command (on one line):
\begin{verbatim}
  svn propset svn:keywords 'LastChangedBy LastChangedDate
  LastChangedRevision HeadURL' filename
\end{verbatim}
Another important keyword is the ignore tag. This is set/changed for
the current directory through the command
\begin{verbatim}
  svn propedit svn:ignore .
\end{verbatim}
and then editing the file names in the editor window. In general,
properties of a specific file or directory are displayed by
\begin{verbatim}
  svn proplist <FILE>
  svn propget <PROPERTY> <FILE>
\end{verbatim}


% --------------------------------------------------------------------
\subsection{Looking inside the repository}
% --------------------------------------------------------------------
Because the \esoft{Subversion} repository is stored in a database file
on a remote server machine you cannot look at the repository files
using Unix/Linux \ecommand{ls} command. Instead you must use the
\esoft{Subversion} subcommand \ecommand{ls}.
\begin{verbatim}
  svn ls https://www2.mech.kth.se/svn/simson/releases
\end{verbatim}
Note that many \esoft{Subversion} subcommands that take a working copy
directory as an argument may also take a repository URL as an
argument.


% --------------------------------------------------------------------
\subsection{Exporting source code}
% --------------------------------------------------------------------
To export the source code from a \esoft{Subversion}
working copy into a source code tree without the \efile{.svn}
directories there is a convenient command:
\begin{verbatim}
  svn export mydirectory simson-export
\end{verbatim}
This creates a new directory \efile{simson-export} with your source
code in it. The only thing \ecommand{export} does is to copy all files
recursively from your working copy (or from a repository URL) into a
new directory, omitting all \efile{.svn} directories.


% ====================================================================
\section{Manually merging a conflict}
% ====================================================================
When \ecommand{svn status} reports a file or directory with a ``C'' in
front of the filename this means that there is a conflict that
\esoft{Subversion} cannot resolve. This could happen if you and
another developer have simultaneously changed the same lines in the
same file. When the other developer commits his changes and you update
your working copy \esoft{Subversion} finds the conflict which you have
to resolve manually.

To resolve a conflict simply edit the conflicting file and then save
it. To help you out, \esoft{Subversion} has written both your
modifications and the modifications from the repository into the
file. In addition, \esoft{Subversion} has saved your file and the
repository's file in your working copy directory (\efile{.mine} and
\efile{.Rxxx} respectively). Note that you need to remove the two
additional files before you can commit your file.


% ====================================================================
\section{Branches}
% ====================================================================
Branches are versions of any file or directory that were forked from
another version of the same file or directory and are subsequently
developed further independent of the original version. A branch can
\emph{e.g.}\ include additional features that are not (yet) meant to be part
of the original file/directory version. Branches are kept in the
directory \epath{branches}. To create a branch, simply copy the base
version to the branch directory, \emph{e.g.}\ from the \efile{trunk/bla}
directory
\begin{verbatim}
  svn copy trunk/bla branches/bla-branch
\end{verbatim}
followed by a commit of the new branch via
\begin{verbatim}
  svn commit trunk/bla-branch -m'Created new branch bla-branch, based on revision XXX'
\end{verbatim}
It is very important to state in the commit comment from which
revision the branch was created. \esoft{Subversion} will not keep
track of that essential information. Note that the commit will create
a new revision which will overwrite the revision information in the
newly copied files.

The main advantage of using branches comes from the fact that it is
possible to keep a branch copy updated with respect to the original
version (\emph{i.e.}\ the one the branch was based upon). To update a branch,
use
\begin{verbatim}
  svn merge -rXXX:YYY trunk/bla/bla.f branches/bla-branch/bla.f
\end{verbatim}
The two revision \ecommand{XXX} and \ecommand{YYY} are essential: They
specify that all the changes in \efile{trunk/bla/bla.f} made between
revisions \ecommand{XXX} and \ecommand{YYY} should be included in
\efile{branches/bla-branch/bla.f}. A merge will probably create some
conflicts which have to be resolved manually. As mentioned above, when
committing the merged version, it is essential to include in the
comment information about the revision numbers the merge was based
on. \esoft{Subversion} will not keep track of this
information. Therefore, use a comment similar to
\begin{verbatim}
  svn commit -m'merged <files> with trunk revisions from XXX to YYY' <files>
\end{verbatim}

In this way, for a subsequent merge, the logs can be used to determine
up to which revision number a merge has already been performed.


% ====================================================================
\section{Private configuration file}
% ====================================================================
After running \ecommand{svn} for the first time you will have a
directory in your home directory named \efile{.subversion}. This
directory contains the file \efile{.subversion/config} can be edited
to customize \esoft{Subversion} behavior.


% ####################################################################
\chapter{Examples}
% ####################################################################
Below follows common examples of \esoft{Subversion} subcommands.

Create a new working copy root directory
\begin{verbatim}
  svn checkout https://www2.mech.kth.se/svn/simson/trunk Simson
\end{verbatim}
List your modifications to the working copy.
\begin{verbatim}
  svn status somedirectory
\end{verbatim}
List modifications to the repository and predict what
\ecommand{update} would do.
\begin{verbatim}
  svn status -u somedirectory
\end{verbatim}
Update your working copy with modifications to the repository.
\begin{verbatim}
  svn update somefile.f90
\end{verbatim}
Update the repository with modifications to your working copy of a
file.
\begin{verbatim}
  svn commit -m 'Your log-message here' somefile.f90
\end{verbatim}
Get an old version of a file from the repository.
\begin{verbatim}
  svn update -r '{20041006}' somefile.f90
\end{verbatim}
Undo modifications to the working copy.
\begin{verbatim}
  svn revert somefile.f90
\end{verbatim}
List history of a file.
\begin{verbatim}
  svn log somefile.f90
\end{verbatim}
List details of your modifications to a working copy file.
\begin{verbatim}
  svn diff somefile.f90
\end{verbatim}
List differences between your working copy and an old repository
version of a file.
\begin{verbatim}
  svn diff -r '{20041006}' somefile.f90
\end{verbatim}
List differences between two old repository versions of a file.
\begin{verbatim}
  svn diff -r '{20041006}:{20040930}' somefile.f90
\end{verbatim}
List differences between two old repository version r110 and r120 of a
file.
\begin{verbatim}
  svn diff -r 110:120 somefile.f90
\end{verbatim}
List files in the repository.
\begin{verbatim}
  svn ls https://www2.mech.kth.se/svn/simson/
\end{verbatim}
Rename a file/directory.
\begin{verbatim}
  svn mv somefile.f90 newname.f90
\end{verbatim}
Put a new file/directory under version control.
\begin{verbatim}
  svn add somefile.f90
\end{verbatim}
Remove a file/directory.
\begin{verbatim}
  svn delete somefile.f90
\end{verbatim}
Export a source code tree for distribution to a user/customer.
\begin{verbatim}
  svn export https://www2.mech.kth.se/svn/simson/releases/mydir mydir
\end{verbatim}
View who has written specific parts of a file
\begin{verbatim}
  svn blame somefile.f90
\end{verbatim}
To resurrect an earlier deleted file (from revison 972)
\begin{verbatim}
  svn cp -r972 https://www2.mech.kth.se/svn/simson/trunk/myfile myfile
\end{verbatim}
To merge changes in trunk version of bla from revision 687 to 691 to
one branch copy of bla
\begin{verbatim}
  svn merge -r687:691 bla/bla.f ../branches/bla2dparallel/bla.f
\end{verbatim}
And finally to get help
\begin{verbatim}
  svn help
\end{verbatim}

\newpage

\bibliographystyle{simson}
\bibliography{fluids}

\end{document}
